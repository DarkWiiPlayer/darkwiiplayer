\documentclass{article}

\usepackage{listings}
\usepackage{color}
\usepackage{graphicx}
\usepackage[backend=biber,style=alphabetic]{biblatex}
\addbibresource{$HOME/me/bibliotheca.bib}

\def\code#1{\texttt{#1}}

\title{Programming is fun!}
\author{DarkWiiPlayer}

\definecolor{listingbg}{rgb}{1,0.96,0.7}

\begin{document}
\lstset{
	backgroundcolor=\color{listingbg},
	tabsize=3, numbers=left,
}

\maketitle

Listings can be done fairly easily in LaTeX; one only has to include the \code{listings} package and start a new \code{lstlisting} block.

\lstset{language=[5.1]Lua,caption={A hello world program in Lua}}
\begin{lstlisting}
-- Comment
print "Hello World!"
\end{lstlisting}

\lstset{language=C,caption={The same program in C}}
\begin{lstlisting}
#include <stdio.h>
// Comment
int main() { printf("Hello World!\n") }
\end{lstlisting}

\begin{figure}[h]
	\centering
	\includegraphics{$HOME/image.png}
	\caption{An image}
	\label{fig:image1}
\end{figure}

Programming is fun! \cite{sicp}

\printbibliography
\end{document}
